%!TEX root = bash.tex
\subsection*{File system}
\begin{enumx}
	\item [\cmd] \textbf{cat}* concatenates and prints files,
	
	\item [\cmd] \textbf{tac}* does the same in reverse: % tac -r -s 'x|[^x]'
	\item [\texttt{A}] shows nonprinting characters,
	\item [\texttt{b}] numbers nonempty output lines,
	\item [\texttt{n}] numbers all output lines,
	\item [\texttt{s}] suppresses repeated empty output lines.
	
	\item [\cmd] \textbf{rev} reverses lines characterwise.
\end{enumx}

\begin{enumx}
	\item [\cmd] \textbf{chgrp} changes group ownership.
	
	\item [\cmd] \textbf{chmod} changes permissions of a file:
	\item [\texttt{ugoa}] permissions of the owner, group, other/all users,
	\item [\texttt{+-=}] adds, removes or sets selected file mode bits,
	\item [\texttt{rwx}] selects file mode bits: read/write/execute (4-2-1).
	
	\item [\cmd] \textbf{chown} changes owner of a file.
	
	\item [\cmd] \textbf{umask} set file mode creation mask.
\end{enumx}

\begin{enumx}
	\item [\cmd] See also \textbf{cksum} (CRC checksums) and \textbf{md5sum}.
	
	\item [\cmd] \textbf{shasum} prints or checks SHA message digests:
	\item [\texttt{a}] algorithm: 1, 224, 256, 384, 512, 512224 or 512256,
	\item [\texttt{b}] reads in binary mode,
	\item [\texttt{c}] checks SHA sums read from the ,,files''.
\end{enumx}
%\textbf{cmp} compares two files byte by byte.

\textbf{cp}* copies files and directories:
\begin{enumx}
	\item [\texttt{a}] never follows symlinks, preserves all attributes,
	\item [\texttt{b}] makes a backup of each existing destination file,
	\item [\texttt{d}] nevers follows symlinks in ,,source'',
	\item [\texttt{f}] removes an existing destination file if needed,
	\item [\texttt{i}] prompts before overwrite,
	\item [\texttt{r}] copies directories recursively,
	\item [\texttt{l}] hard links files instead,
	\item [\texttt{s}] makes symbolic links instead,
	\item [\texttt{t}] copies all ,,source'' arguments into ,,directory'',
	\item [\texttt{u}] copies only when the source file
	 is newer than the destination file.
\end{enumx}

\textbf{dd if=file of=file bs=bytes count=n} converts and copies a file:
\begin{enumx}
	\item [\texttt{if}] reads from a file instead of stdin
	\item [\texttt{of}] writes to a file instead of stdout
	\item [\texttt{bs}] reads and writes up to ,,bytes'' bytes at a time
	\item [\texttt{count}] copies only ,,n'' input blocks
\end{enumx}

\textbf{df} reports file system disk space usage.

\textbf{du}* estimates file space usage:
\begin{enumx}
	\item [\texttt{a}] writes counts for all files, not just directories,
	\item [\texttt{h}] prints sizes in human readable format,
	\item [\texttt{s}] diplays only a total.
\end{enumx}

\textbf{file} determines file type.

\textbf{fsck} checks and repairs a Linux filesystem.

\textbf{fuser} identifies processes using files or sockets.

\textbf{ln}* makes hard links between files
(only in the same file system, does not work with directories):
\texttt{ln -s} makes symbolic links instead.

\textbf{ls} lists directory contents:
\begin{enumx}
	\item [\texttt{a}] does not ignore entries starting with dot
	\item [\texttt{F}] appends indicator to entries
	\item [\texttt{h}] prints human readable sizes
	\item [\texttt{i}] prints the index number of each file
	\item [\texttt{l}] prints permissions, number of hard links, owner, group, size, last-modified date as well.
	\item [\texttt{r}] reverses order while sorting
	\item [\texttt{R}] lists subdirectories recursively
	\item [\texttt{S}] sorts by file size (largest first)
	\item [\texttt{t}] sorts by modification time (newest first)
\end{enumx}

\manualbreak

\textbf{mkdir} makes directories (\texttt{mkdir p}: with parents as needed, no
error if existing).

\textbf{mount} mounts a filesystem.

\textbf{mv}* moves (renames) files:
\begin{enumx}
	\item [\texttt{f}] does not prompt before overwriting,
	\item [\texttt{i}] prompts before overwriting.
\end{enumx}

\textbf{pwd}* prints name of current directory.

\textbf{rm}* removes files or directories:
\begin{enumx}
	\item [\texttt{f}] never prompts,
	\item [\texttt{i}] always prompts,
	\item [\texttt{r}] removes directories and their contents.
\end{enumx}

% \textbf{rmdir} removes (empty) directories.

\textbf{split}* splits a file into pieces:
\begin{enumx}
	\item [\texttt{b}] puts ,,size'' bytes per output file,
	\item [\texttt{n}] generates ,,chunks'' output files.
\end{enumx}

\textbf{tar} stores and extracts files from a tape or disk archive.
\begin{enumx}
	\item [\texttt{c}] creates a new archive,
	\item [\texttt{x}] extracts files from an archive,
	\item [\texttt{t}] lists the contents of an archive,
	\item [\texttt{f}] uses archive file or device
	\item [\texttt{z}] uses zip/gzip
	\item [\texttt{j}] bzip2 compression
	\item [\texttt{k}] does not replace existing files when extracting
\end{enumx}

\textbf{tee} (named after the T-splitter used in plumbing) duplicates pipe content:
\begin{enumx}
	\item [\texttt{a}] appends to the given files, does not overwrite,
	\item [\texttt{i}] ignores interrupts.
\end{enumx}

\textbf{touch} changes file timestamps.

% type

\textbf{umask} set file mode creation mask.
 