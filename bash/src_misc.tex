\subsection{Miscellaneous}
\textbf{bc} is an arbitrary precision calculator language.
\begin{enumx}
	\item \texttt{echo 'obase=16;255' | bc} prints \texttt{FF},
	\item \texttt{echo 'ibase=2;obase=A;10' | bc} prints \texttt{2},
	\item \texttt{scale=10} (after \texttt{bc -l}) sets working precision.
\end{enumx}

\textbf{dc} is a reverse-polish desk calculator.
One of the oldest Unix utilities, 
predating even the invention of the C programming language.

\textbf{cal}, \textbf{ncal} displays a calendar and the date of Easter.
\begin{enumx}
	\item [\texttt{e}] displays date of Easter,
	\item [\texttt{j}] displays Julian days,
	\item [\texttt{m}] displays the specified month,
	\item [\texttt{w}] prints the numbers of the weeks,
	\item [\texttt{y}] displays a calendar for the specified year,
	\item [\texttt{3}] displays the previous, current and next month.
\end{enumx}

\textbf{date} prints or set the system date and time.

% \textbf{expr}

\textbf{lp} prints files.

\textbf{od} dumps files in octal.

\textbf{sleep} delays for a specified amount of time.

\textbf{true}, \textbf{false} does nothing, (un)successfully.

\manualbreak