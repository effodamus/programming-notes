\subsection{User environment}
\textbf{clear} clears the terminal screen. \hfill (\emph{Ctrl-L}) 

\textbf{env} runs a program in a modified environment.

\textbf{exit} terminates the calling process.

\textbf{finger} is a user information lookup program. 

\textbf{history} displays the history list with line numbers.

\textbf{logname} prints user's login name.

\textbf{mesg} displays (or does not display) messages from other users.

\manualbreak

\textbf{passwd} changes user password:
\begin{enumx}
	\item [\texttt{d}] deletes an account's password (makes it empty),
	\item [\texttt{e}] expires an account's password,
	\item [\texttt{n}] sets the minimum number of days between password changes,
	\item [\texttt{w}] sets the number of days of warning 
	before a password change is required,
	\item [\texttt{x}] sets the maximum number of days a password remains valid.
\end{enumx}

\textbf{su} changes user ID or becomes superuser.

\textbf{sudo} executes a command as another user.

%\textbf{talk}

\textbf{tput} initializes a terminal or queries terminfo database.

\textbf{uname} prints system information:
\begin{enumx}
	\item [\texttt{a}] all information, in the following order:
	\item [\texttt{s}] the kernel name,
	\item [\texttt{n}] the network node hostname,
	\item [\texttt{r}] the kernel release,
	\item [\texttt{v}] the kernel version,
	\item [\texttt{m}] the machine hardware name,
	\item [\texttt{p}] the processor type,
	\item [\texttt{i}] the hardware platform,
	\item [\texttt{o}] the operating system.
\end{enumx}

\textbf{uptime} tells how long the system has been running.

\textbf{wall} writes a message to all users,
\textbf{write} sends a message to another user. 

\textbf{who} shows who is logged on,
\textbf{w} shows who is logged on and what they are doing,
\textbf{whoami} prints effective userid.
