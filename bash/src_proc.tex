%!TEX root = bash.tex
\subsection*{Processes}

\begin{enumx}
	\item [\cmd] \textbf{chroot} changes the root directory 
	for the current running process and their children.
\end{enumx}

\begin{enumx}
	\item [\cmd] \textbf{at} schedules commands to be executed once, 
	at a particular time in the future: it accepts times of the form 
	\texttt{HH:MM}, \texttt{midnight}, \texttt{noon} or \texttt{teatime}; 
	\texttt{MMDD[CC]YY}, \texttt{MM/DD/[CC]YY}, \texttt{DD.MM.[CC]YY} or 
	\texttt{[CC]YY-MM-DD} (the specification of a date 
	must follow the specification of the time of day).
	You can also give times like \texttt{now + 3 hours}.
\end{enumx}

\begin{enumx}
	\item [\cmd] \textbf{bg} resumes suspended jobs in the background.
	\item [\cmd] \textbf{fg} resumes suspended jobs in the foreground.
	\item [\cmd] \textbf{jobs} lists the active jobs.
\end{enumx}

\begin{enumx}
	\item [\cmd] \textbf{cron}: a daemon executing scheduled commands.
	\item [\cmd] \textbf{crontab} maintain individual users' crontab files.
\end{enumx}

\begin{enumx}
	\item [\cmd] \textbf{kill} sends a \texttt{TERM} signal to a process.
	\item [\cmd] \textbf{killall} kills processes by name.
\end{enumx}

\begin{enumx}
	\item [\cmd] \textbf{ps} reports a snapshot of the current processes.
	\item [\cmd] \textbf{pstree} displays a tree of processes.
\end{enumx}

\begin{enumx}
	\item [\cmd] \textbf{nice} changes process priority.
\end{enumx}

\begin{enumx}
	\item [\cmd] \textbf{pgrep}, \textbf{pkill} looks up or signals 
processes based on name and other attributes.
\end{enumx}

\begin{enumx}
	\item [\cmd] \textbf{time} runs programs and summarizes system resource usage. 
\end{enumx}

\begin{enumx}
	\item [\cmd] \textbf{top} displays linux processes.
\end{enumx}