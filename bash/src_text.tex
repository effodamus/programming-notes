\subsection{Text processing}
% \textbf{aspell} is an interactive spell checker.

\textbf{awk} is a pattern scanning and processing language.

%\textbf{banner}

\textbf{basename} strips directory and suffix from filenames, 
\textbf{dirname} strip last component from file name.

\textbf{comm} compares two sorted files line by line.

\textbf{csplit} splits a file into sections determined by context lines.

\textbf{join} joins lines of two files on a common field.
\textbf{paste} merges lines of files.
\textbf{cut}* removes sections from each line of files:
\begin{enumx}
	\item [\texttt{d}] uses ,,delim'' instead of Tab for field delimeter,
	\item [\texttt{f}] selects only these fields.
\end{enumx}

\textbf{diff} compares files line by line.

% \textbf{ed}
% \textbf{ex}

\textbf{fmt} is a simple optimal text formatter, 
\textbf{fold} wraps each input line to fit in specified width.

\textbf{head}* outputs the first part of files:
\begin{enumx}
	\item [\texttt{c}] the first ,,num'' bytes,
	\item [\texttt{n}] the first ,,num'' lines.
\end{enumx}

\textbf{iconv} converts text from one character encoding to another.


\textbf{less} is opposite of \textbf{more}, a file perusal filter for crt viewing.

\textbf{nl*} numbers lines of files:
\begin{enumx}
	\item [\texttt{s}] adds ,,string'' after line number,
	\item [\texttt{w}] uses ,,number'' columns for line numbers.
\end{enumx}


\textbf{printf} formats and prints data.

\textbf{sed} is a stream editor for filtering and transforming text.

\textbf{shuf*} generates random permutations:
\begin{enumx}
	\item [\texttt{e}] treats each ,,arg'' as an input line,
	\item [\texttt{i}] treats each number .. through .. as an input line, 
	\item [\texttt{n}] outputs at most ,,count'' lines,
	\item [\texttt{r}] output lines can be repeated (with \texttt{-n}).
\end{enumx}

\textbf{sort}* sorts lines of text files:
\begin{enumx}
	\item [\texttt{g}] compares general numerical values,
	\item [\texttt{h}] compares human readable numbers,
	\item [\texttt{n}] compares string numerical values,
	\item [\texttt{r}] reverses the results.
\end{enumx}

\textbf{strings} prints the strings of printable characters in files.

\textbf{tail}* outputs the last part of files:
\begin{enumx}
	\item [\texttt{c}] the last ,,num'' bytes,
	\item [\texttt{f}] outputs appended data as the file grows,
	\item [\texttt{n}] the last ,,num'' lines.
\end{enumx}

\textbf{tr}* translates or deletes characters:
\begin{enumx}
	\item \texttt{tr abc xyz} changes \texttt{a} to \texttt{x}, $\ldots$,
	\item [c] uses the complement of ,,set1'',
	\item [d] deletes characters, does not translate,
	\item [s] replaces each sequence of a repeated character that is listed 
	in the last specified ,,set'' with a single occurrence of that character.
\end{enumx}

\textbf{uniq*} omits repeated lines:
\begin{enumx}
	\item [\texttt{c}] prefixes lines by the number of occurences
	\item [\texttt{d}] only prints duplicate lines, one for each group
	\item [\texttt{f}] avoids comparing first fields
	\item [\texttt{i}] ignores differences in case
	\item [\texttt{s}] avoids comparing first characters
	\item [\texttt{w}] compares no more than $n$ characters
\end{enumx}

\textbf{vim} a programmers text editor.

\textbf{wc}* prints newline, word and byte counts:
\begin{enumx}
	\item [\texttt{c}] prints the byte counts,
	\item [\texttt{l}] prints the newline counts,
	\item [\texttt{m}] prints the character counts,
	\item [\texttt{w}] prints the word counts.
\end{enumx}

\textbf{xargs} builds and executes command lines from standard input.

\textbf{yes} outputs a string repeatedly until killed.
