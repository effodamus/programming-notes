\subsection{Networking}
\textbf{curl} transfers a URL.

\textbf{dig} is a DNS lookup utility (domain information groper).
\begin{enumx}
	\item [\texttt{x}] simplified reverse lookups.
\end{enumx}

\textbf{host} is a DNS lookup utility.

\textbf{ifconfig} configures a network interface.

\textbf{inetd} is a super-server daemon that provides Internet services.

\textbf{netcat}: arbitrary TCP and UDP connections and listens.

\textbf{netstat} prints network connections, routing tables, 
interface statistics, masquerade connections, and multicast memberships.

\textbf{nslookup} queries Internet name servers interactively.

\textbf{ping} tests the reachability of a host 
on an IP network by sending ICMP ECHO\_REQUEST:
\begin{enumx}
	\item [\texttt{c}] stops after sending ,,count'' packets,
	\item [\texttt{n}] numeric output only, 
	avoids to lookup symbolic names for host addresses. 
\end{enumx}

\textbf{rdate} sets the system's date from a remote host.

\textbf{rlogin} is an OpenSSH SSH client (remote login program)

\textbf{route} shows and manipulates the IP routing table.

\textbf{ssh} is an OpenSSH SSH client (remote login program).
\begin{enumx}
	\item [\texttt{D}]
	\item [\texttt{p}]
	\item [\texttt{X}]
\end{enumx}

\textbf{traceroute} is a computer network diagnostic tool for 
displaying the route (path) and measuring transit delays of 
packets across an Internet Protocol (IP) network.

\textbf{wget} is a non-interactive network downloader.
\begin{enumx}
	\item [\texttt{A}, \texttt{R}] specifies lists 	of file suffixes or 
	patterns (when wildcard characters appear) to accept or reject,
	\item [\texttt{b}] goes to background immediately after startup,
	\item [\texttt{c}] continues getting a partially-downloaded file,
	\item [\texttt{m}] turns on options suitable for mirroring: 
	infinite recursion and time-stamping,
	\item [\texttt{np}] does not ever ascend to the
	parent directory when retrieving recursively,
	\item [\texttt{U}] identifies as ,,agent-string'' to the HTTP server.
	\item [\texttt{w}] waits the specified number of seconds 
	between the retrievals (see also \texttt{--random-wait}).
\end{enumx}